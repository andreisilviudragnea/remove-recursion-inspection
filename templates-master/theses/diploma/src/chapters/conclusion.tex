\chapter{Conclusion}

The algorithm for transforming recursion in iteration is composed of thirteen passes which gradually transform the body
of the method. The end result is a method which is functionally equivalent to the initial method, but the recursion does
not take place implicitly anymore, but explicitly by simulating it in the user program with a stack of frames. This is
how \code{StackOverflowError}s can be avoided, at the cost of obfuscating the code.

The performance of the generated code is generally worse than the performance of the original code, but this is
generally acceptable, because this refactoring should only be used as a temporary solution to
\code{StackOverflowError}s.